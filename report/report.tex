\documentclass[article]{amsart}
\usepackage[margin=2cm]{geometry}
\usepackage{graphicx}
\usepackage{enumitem}
\setlist[description]{}
\usepackage{hyperref}
\usepackage{fancyhdr}
\usepackage[utf8]{inputenc}

\hypersetup{
  colorlinks=true,
  linkcolor=blue,
  urlcolor=cyan,
  linktoc=all
}


\pagestyle{fancy}
\fancyhf{}
\rhead{\thepage}

\begin{document}
\thispagestyle{empty}
\title{
	Simulaci\'on de eventos discretos \\
	n servidores en paralelo
}
\author{Chavely Gonz\'alez Acosta C412}
\date{\today}
\maketitle
\pagenumbering{gobble}

\newpage

\pagenumbering{arabic}

\tableofcontents

\pagestyle{fancy}
\newpage

\section{Introducci\'on}

\begin{enumerate}
\item \textbf{Breve descripción del proyecto} \\
El sistema a simular en el presente proyecto consiste en: m clientes, los cuales llegan a un sistema que tiene n servidores, con tiempo entre los arribos que distribuye M. Cuando un cliente llega, se une a la cola si ambos servidores est\'an ocupados. El caso de solo tener dos servidores se describe a continuaci\'on: Si el servidor 1 est\'a libre, el cliente entra en servicio con el servidor 1. Si el servidor 1 está ocupado pero el servidor 2 está libre, el cliente entra en servicio con el servidor 2. Cuando un cliente completa el servicio con un servidor (no importa cuál), ese cliente luego abandona el sistema. El cliente que ha estado en la cola durante más tiempo (si hay clientes en la cola) entra en servicio. La distribución de servicio en el servidor i es $ G_{i} $\\
\item \textbf{Objetivos y metas} \\
El objetivo fundamental de este proyecto es simular correctamente el sistema antes descrito y arribar a conclusiones.\\
\item \textbf{Variables que describen el problema} \\
 Las variables de interés para este proyecto incluyen el tiempo entre los arribos de los clientes, la distribución de los servidores ocupados y la duración del servicio en cada servidor.\\
\end{enumerate}
\newpage
\section{Detalles de implementaci\'on}
La implementación de la simulación de eventos discretos se basa en los siguientes pasos:

    
\begin{enumerate}
\item \textbf{Modelado del sistema}\\
El sistema se compone de:
\begin{itemize}
\item Un conjunto de m clientes que arriban con una diferencia de tiempo que distribulle M y requieren alg\'un servicio\\
\item n servidores: $s_{1}, s_{2}, s_{3}, ..., s_{n-1}, s_{n}$, los cuales trabajan en paralelo y el tiempo que tardan en completar un servicio distribulle $G_{i}$\\
\item Una cola(FIFO) de clientes, inicialmente vac\'ia, para mantener a los clientes en espera mientras est\'an llenos todos los servidores hasta que se vac\'ie alg\'un servidor\\
\item En todo momento el servidor al cual se le asigna un servicio nuevo es, de los que no est\'en ocupados en ese momento, el de menor \'indice.\\
\end{itemize}

\item \textbf{Generaci\'on de eventos}\\
Para generar las variables aleatorias se utilizan los m\'etodos del m\'odulo de python $numpy.random$. Las distribuciones propuestas son las siguientes: poisson, geom\'etrica, uniforme, binomial, exponencial, normal, beta y ganma, con sus respectivos par\'ametros predefinidos por el propio programa. Al momento de relizar la simulaci\'on se decide cual(es) utilizar.\\
\begin{itemize}
\item Variables aleatorias: distribución de llegada de los clientes (M) y las distribuciones de servicio en los servidores ($G_{i}$)\\
\end{itemize}
\item \textbf{Procesamiento de eventos}\\
Para simular los eventos: llegada de un cliente, agregar un cliente a la cola, sacar a un cliente de la cola, asignar un servidor para un servicio, ejecutar un servicio en un servidor y liberar un servidor al terminar de realizar un servicio; se utilizaron las siguientes clases: \\
\begin{itemize}
\item heap: una cola con prioridad, que permite eliminar($O(log(n))$) e insertar($O(log(n))$) elementos manteniendo siempre en el tope el elemento de menor valor($O(1)$). Esta estructura se utilizo para mantener el conjunto de servidores disponibles de forma tal que siempre que se pide(elimina) uno devuelve el de menor indice, y para mantener el conjunto de servidores ocupados, de forma tal que siempre que se pide(elimina) uno devuelve el que termina primero(cuyo tiempo de finalizaci\'on es menor). \\
\item queue: una cola que cumple con la invariante: el primer elemento que entra es el primero que sale(FIFO). Esta estructura se utiliza para mantener a los clientes en espera ordenados mientras no hay servidores vac\'ios. Pero como sabemos que estos elementos siempres ser\'an enteros consecutivos basta saber cuantos hay y el valor de el primero para poder implementar los m\'etodos pop(elimina el primero), push(inserta un elemento al final) y empty(dice si esta vac\'iao no la cola)\\
\item action: un objeto con 4 propiedades: momento, nombre, servidor y cliente, el cual permite almacenar la informaci\'on sobre un evento dado.\\
\item report: un objeto que permite almacenar toda la informaci\'on que ocurre durante una simulaci\'on mediante una lista de accionesy luego procesarla.\\
\item simulation: se encarga de realizar la simulaci\'on, recibe como par\'ametros las distribuciones que siguen los servidores y el tiempo entre la aparici\'on de clientes. Para esto utiliza la siguiente l\'ogica: se agregan todos los clientes de uno en uno, al momento de agregarlos primero se genera el tiempo que demora en aparecer el cliente, y se actualiza el curent time de la simulaci\'on, luego se ejecutan todos los servicios posibles hasta ese momento, es decir, todos los servidores que estuviesen ocupados en alg\'un servicio cuyo tiempo de finalizacion fuese menor que el momento actual se eliminan del heap de los servidores ocupados y se agregan al heap de los servidores activos, a la vez si hab\'ian clientes en la cola se van procesando al momento en que se vac\'ia alg\'un servidor, finalmente cuando ya no hay m\'as servidores que terminen antes del momento actual se procesa el nuevo cliente, es decir, se escoge el servidor con menor \'indice que se encuentre desocupado, se elimina de los servidores activos, se genera un tiempo aleatorio que se demorar\'a en realizar el servicio y se agrega a los servidores ocupados con un tiempo de finalizaci\'on igual al momento actual m\'as el tiempo generado, de no haber servidores activos se agrega el cliente a la cola. En todo momento se agrega la acci\'on que se est\'a realizando al $simulation.report$. Finalmente se procesan todos los clientes que no han teminado de ejecutar sus respectivos servicios o est\'an en cola, se imprimen los resultados y se reinicia elestado de la simulaci\'on.\\
\end{itemize}
\item \textbf{Simulaci\'on}\\
Ejecutar la simulación para un número determinado de eventos o hasta que se cumpla una condición de parada.\\
\item \textbf{Análisis de resultados}\\
Recopilar y analizar los datos de la simulación para entender el comportamiento del sistema y extraer insights.\\
\end{enumerate}

\newpage
\section{Resultados y experimentos}

Los resultados de la simulación proporcionarán información valiosa sobre el comportamiento del sistema bajo condiciones específicas. Esto incluirá:

    Hallazgos de la simulación: Observaciones sobre la utilización de los servidores, la duración de las colas y los tiempos de servicio.

    Interpretación de los resultados: Análisis de cómo los resultados se relacionan con las variables de interés y las expectativas del modelo.

    Análisis estadístico: Evaluación de las variables de interés mediante análisis estadístico para entender su comportamiento y variabilidad.

    Análisis de parada de la simulación: Identificación de las condiciones que se utilizaron para detener la simulación y su impacto en los resultados.

\newpage
\section{Modelo matem\'atico}

\begin{enumerate}
\item \textbf{Descripción del modelo de simulación}\\
    La llegada de un cliente se simula tomando el momento de llegada del \'ultimo cliente, se genera un entero que distribuya M, se actualiza el estado de la simulaci\'on hasta el momento de llegada del \'ultimo cliente m\'as el tiempo que tard\'o el cliente actual en llegar y si no hay servidores activos se agrega el nuevo cliente a la cola, si en cambio hay alg\'un servidor activo se le asigna al cliente actual.\\
    La asignaci\'on de un servidor a un cliente consiste en eliminar el servidor de los servidores activos, y agregarlo junto con el momento en el cual terminar\'a de realizar el servicio al conjunto de los servidores ocupados. El momento en el cual terminar\'a de realizar el servicio es: momento actual m\'as n\'umero aleatorio generado seg\'un la distribuci\'on del servidor.\\
    La salida de os clientes se realiza cuando se actualiza el estado de la simulaci\'on tras actualizar el momento actual de la misma. Se comprueba si el momento de terminar el servicio ya pas\'o y se elimina el servidor del conjunto de servidores ocupados y se agrega al conjunto de servidores activos.\\

\item \textbf{Supuestos y restricciones}\\
Definición de los supuestos del modelo, como la independencia de los eventos de llegada y la asincronicidad de los servicios.


\end{enumerate}

\end{document}